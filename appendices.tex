\documentclass[dissertation.tex]{subfiles}

\begin{document}
\begin{appendices}
\chapter{Implementation of \haskell{Integer}}\label{appendix:integer-literal}
{
    Here is an example implementation of \haskell{Integer}, Haskell's arbitrary precision integral value type: it is essentially a wrapper around Java's \java{BigInteger} class. The copious uses of underscores is explained in Section \ref{sec:jvm-sanitisation}.

    \begin{javafigure}
    import java.math.BigInteger;

    public class _Integer extends Data {
        public BigInteger value;
        public static _Integer _make_Integer(BigInteger x) {
            _Integer i = new _Integer();
            i.value = x;
            return i;
        }
        public static _Integer _make_Integer(String x) {
            return _make_Integer(new BigInteger(x));
        }

        public static _Integer add(_Integer x, _Integer y) {
            return _make_Integer(x.value.add(y.value));
        }
        ... // Analogous functions for subtraction and multiplication

        public static boolean eq(_Integer x, _Integer y) { ... }

        public static String show(_Integer x) { ... }
    }
    \end{javafigure}

    The \java{_make_Integer(String)} function is used by the compiler to construct \haskell{Integer} literals: it allows a Java \java{_Integer} object to be constructed from a Java string representation. For example, the bytecode that creates the Haskell literal \haskell{2} would load the string \monospace{"2"} from the constant pool then invoke the creation method:

    \begin{monospacefigure}
    ldc          #210       // String 2
    invokestatic #16        // Method tmp/_Integer._make_Integer:(Ljava/lang/String;)Ltmp/_Integer;
    \end{monospacefigure}

    The \java{add}, \java{eq}, etc. methods are Java implementations of the functions required by Haskell's \haskell{Num}, \haskell{Eq} and \haskell{Show} typeclass instances for \haskell{Integer}. For `builtin' types, the implementation of these typeclass functions need to be given in Java, as they cannot be expressed in Haskell. Section \ref{sec:hooks} on Hooks covers this aspect of code generation in more detail.
}
\chapter{The \java{Function} class}\label{appendix:function}
{
    \begin{javafigure}
import java.util.ArrayList;
import java.util.function.BiFunction;

public class Function extends HeapObject {
    private BiFunction<HeapObject[], HeapObject[], HeapObject> inner;
    private HeapObject[] freeVariables;
    private ArrayList<HeapObject> arguments;
    private int arity = 0;
    private HeapObject result = null;

    public Function(BiFunction<HeapObject[], HeapObject[], HeapObject> inner, int arity, HeapObject[] freeVariables) {
        this.inner = inner;
        this.arity = arity;
        this.freeVariables = freeVariables;
        arguments = new ArrayList<>();
    }

    @Override
    public HeapObject enter() {
        // Check if we've got a cached value
        if (result != null) {
            return result;
        }

        if (arguments.size() < arity) {
            return this;
        }
        else if (arguments.size() > arity) {
            try {
                Function fun = (Function)inner
                    .apply(arguments.subList(0, arity).toArray(new HeapObject[0]), freeVariables)
                    .enter()
                    .clone();
                for (HeapObject arg : arguments.subList(arity, arguments.size()))
                    fun.addArgument(arg);
                result = fun.enter();
                return result;
            }
            catch (CloneNotSupportedException e) {
                throw new RuntimeException(e);
            }
        }
        else {
            result = inner.apply(arguments.toArray(new HeapObject[0]), freeVariables).enter();
            return result;
        }
    }

    public void addArgument(HeapObject arg) {
        arguments.add(arg);
    }

    @Override
    public Object clone() throws CloneNotSupportedException {
        Function f = (Function)super.clone();
        f.inner = inner;
        f.arity = arity;
        f.freeVariables = freeVariables.clone();
        f.arguments = new ArrayList<>(arguments);
        return f;
    }
}
    \end{javafigure}
}
\chapter{Repository Structure}\label{appendix:directory-structure}
{
    The top-level repository structure is within 4 directories:

    \footnotesize
    \directorystructure{
    [./
        [app]
        [benchmarks]
        [src]
        [test]
    ]
    }

    \section{\monospace{compiler}}
    {
        \footnotesize
        \directorystructure{
        [src/
            [Compiler.hs]
            [Backend/
                [CodeGen.hs]
                [Deoverload.hs]
                [ILAANF.hs]
                [ILA.hs]
                [ILB.hs]
            ]
            [Optimisations/
                [BindingDedupe.hs]
                [LetLifting.hs]
                [UnreachableCodeElim.hs]
            ]
            [Preprocessor/
                [Renamer.hs]
            ]
            [Typechecker/
                [Typechecker.hs]
            ]
        ]
        }
    }
    \section{\monospace{compiler-exe}}
    {
        \footnotesize
        \directorystructure{
        [app/
            [Main.hs]
        ]
        }
    }
    \section{Tests}
    {
        \footnotesize
        \directorystructure{
        [test/
            [AlphaEqSpec.hs]
            [Backend/
                [DeoverloadSpec.hs]
                [ILAANFSpec.hs]
                [ILASpec.hs]
            ]
            [Preprocessor/
                [DependencySpec.hs]
                [RenamerSpec.hs]
            ]
            [Typechecker/
                [TypecheckerSpec.hs]
            ]
            [Spec.hs]
            [WholeProgram.hs]
        ]
        }
    }
    \section{Benchmarks}
    {
        \footnotesize
        \directorystructure{
        [benchmarks/
            [benchmark.py]
            [etabenchmark.py]
            [fregebenchmark.py]
            [fregec.jar]
            [javabenchmark.py]
            [jhaskellbenchmark.py]
            [jmhbenchmark.py]
            [Main\_Template.java]
            [plot.py]
            [programs
                [ackermann.eta]
                [ackermann.fr]
                [ackermann.hs]
                [factorial.eta]
                [...]
            ]
            [results
                [...]
            ]
            [results.py]
            [runbenchmarks.py]
        ]
        }
    }
    \section{\monospace{hs-java}}
    {
        \footnotesize
        \directorystructure{
        [./
            [JVM/
                [Assembler.hs]
                [Builder/
                    [Instructions.hs]
                    [Monad.hs]
                ]
                [Builder.hs]
                [ClassFile.hs]
                [Common.hs]
                [Converter.hs]
                [Dump.hs]
                [Exceptions.hs]
            ]
            [Java/
                [...]
            ]
        ]
        }
    }
}
\end{appendices}
\end{document}