\documentclass[dissertation.tex]{subfiles}

\begin{document}

Preparation mostly involved reading papers, articles, and compiler documentation to form a rough idea of how to
implement each stage of the compiler. 

% Probably outline the major areas looked into: type system, WHNF, ANF, lazy evaluation.

% TODO(kc506): Explain the scheduling issues here

\section{Testability}
{

    Pure functions are usually easier to unit test than impure functions as behaviour is only affected by the
    parameters, independent of any global mutable state. I wrote the pipeline of the compiler using solely pure code,
    only using impure code for reading the source file and writing the compiled files, which made testing each stage
    reliable and strictly independent of the adjacent pipeline sections.

    Regression tests were implemented for all major bugs discovered, and ensure that the compiler stages don't
    reintroduce incorrect behaviour.

    Finally, end-to-end tests ensure that the compiler successfully processes a given Haskell source file and that the
    executable produced computes the correct result, treating the compiler as a black box. This extremely coarse testing
    method was very effective for discovering the existence of bugs, which could then be tracked down using standard
    debugging techniques and isolated using the finer-grained unit and regression tests.

    \todo[inline]{kc506: check this doesn't repeat stuff from the evaluation chapter}

}
\section{Tools}
{

    I chose Haskell as my implementation language as it has a number of desirable features for large projects: purity
    ensures that components of the project cannot interact in unexpected ways, and the static type system guarantees
    that modifications are checked for a shallow (type-level) degree of correctness across the entire system.

    The natural choice of compiler for Haskell is the industry-leading Glasgow Haskell Compiler (GHC), and the Stack
    build system is also relatively uncontested for Haskell build tooling, ensuring reproducible builds through a strict
    dependency versioning system.

    Documentation has been written using Haddock, a tool that generates documentation from the code and comments: this
    documentation is rebuilt on every successful build and provides an easily-navigable description of commented modules
    and functions.

    Git was used for version control, allowing me to develop features on distinct branches, use bisection to find the
    commits which introduced bugs, and keep a remote repository of code on Github as a backup.

    Continuous integration was performed using Travis CI, which ensures that tests are run on every pushed commit and
    that builds are reproducible: the project can be built and run on different machines.

    The benchmarking framework was written in Python 3, and plots generated using \monospace{matplotlib}.

}
\section{Software Development Model}
{

    I mainly used the waterfall development model, building each stage of the compiler sequentially and testing it both
    in isolation and in sequence with the previous stages. The only stage which broke this model was type inference,
    which required multiple refining iterations to properly implement.

    This approach was effective for most stages as there was a well-defined set of unchanging requirements. When the
    approach failed to work it was due to an incomplete set of requirements, so an iterative approach was more suitable
    to introduce support for the new requirements.

}
\section{Starting Point}
{

    The compiler uses a number of open-source packages from the de-facto Haskell standard library, such as
    \monospace{containers}, \monospace{text}, \monospace{mtl}, \dots. The full list is available in the
    \monospace{packages.yaml} configuration file in the root of the code repository.

    The \monospace{haskell-src} lexing/parsing library for Haskell 98 source code was used, although forked and modified
    minorly (\gitstats{187}{68}). The bytecode assembly library \monospace{hs-java} was forked and significantly
    modified and extended to meet the requirements of this project (\gitstats{1,772}{1,431}).

    No new languages had to be learnt for this project: I was already familiar with Haskell, Python 3, and Java. I had
    not worked with \monospace{matplotlib}, \monospace{haskell-src} or \monospace{hs-java} before, but I was familiar
    with all other tools or libraries mentioned.

}

\end{document}